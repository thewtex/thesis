% preliminary_proposal.tex
%
% This file is root file for Matt McCormick's preliminary proposal
% based on University of Wisconsin-Madison LaTeX Style file withesis
% available in the CTAN



%=====================================================================
% Document Style
%=====================================================================
% Choose only one of the following document classes:
%
% for a 12 Point UW PhD Thesis without Margin Check
% \documentclass[12pt]{withesis}
% the twoside option is for printing on both sides of paper
\documentclass[12pt,twoside]{withesis}
%
% for a 10 Point UW PhD Thesis with Margin Check
%\documentclass[10pt,margincheck]{withesis}
%
% The margincheck option flags lines which overflow their hbox with a black
%  box at the end of the line.  This usually (but not always) indicates a
%  margin violation on the right margin.  Left margin violations aren't
%  indicated and if the margin violation is large enough, there isn't room
%  for the black box to be visiable.  
%
% This option can be also used in conjunction with the msthesis option.
%
% or for a 12 Point UW Masters Thesis
%\documentclass[12pt,msthesis]{withesis}
%
% or for a 10 Point UW Masters Thesis
%\documentclass[10pt,msthesis]{withesis}
%
% The msthesis option changes the page margins from 1" all around
% (the PhD format) to 1.25" left and 1" remaining margins (MS format).
% The defaults for degree and thesis are changed to be MS and thesis.
% These defaults can be overridden if the margins for the MS thesis
% are desired for other documents.

% To include optional packages, use the \usepackage command.
%  The package epsfig is used to bring in the Encapsulated PostScript
%    figures into the document.
%  The package times is used to change the fonts to Times Roman; however
%    because the times typewriter font looks odd, the original LaTeX
%    Computer Modern font is kept for the typewriter font using
%      \renewcommand{\ttdefault}{cmtt}
%    Note that Times Roman is a PostScript font and therefore, the document
%    cannot be correctly viewed from the *.dvi file.  It should be converted
%    to a *.ps file first and then viewed with a PostScript previewer...
\usepackage{times}
\renewcommand{\ttdefault}{cmtt}

%========================================================================
%  Draft Control Commands:
%========================================================================
%
% \psdraft causes the \psfig or \epsfig commands to draw a box and label
% the box with the postscript file name instead of reading in the full
% postscript figure.  This can save time and toner when printing drafts.
%
%\psdraft
%
%
% \psfull causes the inclusion of the postscript figures.
%\psfull
%
%
%\pagestyle{thesisdraft} causes the footer text to become:
% DRAFT: Do Not Distribute        <time><Date>        <input file name>
%
%\pagestyle{thesisdraft}
%
%\pagestyle{thesis} causes the header and footers to be the correct format
%
\pagestyle{thesis}
%
%
%  The page margins can be marked with a post-script box using the
%  \draftmargins command.  This command uses dvips's end-of-page hook
%  This is only visible in the *.ps file (NOT the *.dvi file)!
%
%\draftmargins
%
%
%  The word ``DRAFT'' can be diagonally printed across the page using
%  the \draftscreen command.  This command uses dvip's beginning-of-page
%  hook.  This is only visible in the *.ps file (NOT the *.dvi file)!
%
%\draftscreen


%=======================================================================
% Remove the following lines if appendix tables or figures are present.
% The suppress writing the auxiliary information which appears in the
% list of tables or list of figures.
%
\noappendixtables                % Don't have appendix tables
\noappendixfigures               % Don't have appendix figures


% hyphenation penalty, 50=weak 10000=super-strong
\hyphenpenalty=5000
% how much space is allowed between words
\tolerance=500
% widow penalty, last line of a paragraph at the start of a page
\widowpenalty=500
% orphan penalty, the first line of a paragraph at the beginning of a page
\clubpenalty=500

\usepackage{graphicx}
\usepackage{subfigure}

%=======================================================================
% End of Preamble, start of document
%


\begin{document}

\tableofcontents

\chapter{Specific Aims}


\chapter{Human Health Significance}

\section{Etiology of stroke and the role of atherosclerotic plaque}
\section{Clinical carotid ultrasound and plaque characterization}
\subsection{Defining vulnerable plaque}
\subsection{Plaque characterization with other methods and imaging modalities}
\subsection{Plaque characterization with diagnostic ultrasound}
\section{High frequency ultrasound on carotid plaque}
\section{Transcranial Doppler for monitoring microembolic events}


\chapter{Novel motion estimation with hierarchical Naive Bayesian multilevel
adaptive estimation}

\section{Prior motion tracking algorithms}
\subsection{Kernel based tracking}
\subsection{Multilevel image registration techniques}
\section{The hierarchical Naive Bayesian estimator}
\subsection{Theory}
\subsection{Application in other fields}
\subsection{Application to image motion tracking}
\subsubsection{Hierarchical deconstruction of the image}
\subsubsection{Efficient algorithmic implementation applied to discrete data}
\subsubsection{Regularization techniques}


\chapter{Validation of motion tracking algorithm with simulated and phantom
data}

\section{Simulated data}
\subsection{Mechanical model}
\subsubsection{Spherical inclusion}
\subsubsection{Arterial model}
\subsection{Acoustic model}

\section{Phantom model}

\section{Performance}
\subsection{Accuracy}
\subsection{Precision}
\subsection{Robustness}
\subsection{Stability}
\subsection{Computational efficiency}


\chapter{Application of the novel strain imaging algorithm to \textit{in vivo}
plaques}

\section{Strain images}

\section{Cardiac cycle waveforms}

\section{Combination of normal strains and shear strain into single strain index}
\subsection{Maximum principal strain}
\subsection{Maximum shear strain}
\subsection{Total strain energy}
\subsection{Distortional energy}

\section{Compounding of beam steered strain images}
\subsection{Registration}
\subsection{Strain index consistency}
\subsection{Changes in images and waveforms relative to non-compounded images}


\chapter{High-frequency 3D ultrasound characterization}

\section{Collection and analysis of 3D radiofrequency data}
\subsection{VisualSonics Vevo 770 system}
\subsection{Volume concatenation, storage, and processing}
\subsection{Scan conversion}

\section{Reference phantom development and characterization}
\subsection{Phantom design}
\subsection{Attenuation characterization}
\subsection{Phase velocity characterization}
\subsection{Absolute backscater measurement}

\section{Backscatter analysis}

\section{Attenuation estimation}
\subsection{Water-tissue segmentation}
\subsection{Spectral difference estimation}
\subsection{Spectral shift estimation}
\subsection{Hybrid estimation}


\chapter{Transcranial Doppler detection of microemboli}

\section{Methods to increase robustness of unstable data}
\subsection{Examination room protocol}
\subsection{Post processing software design}

\section{Results}


\chapter{Effectiveness of ultrasonic strain imaging algorithm to detect high
risk plaque}

\section{Histological classification}
\subsection{Registration methods}
\subsection{Strain imaging results}
\subsection{Backscatter results}

\section{Transcranial Doppler}

\section{Symptomatic/Asymptomatic status}

\section{MRI indices of neural atrophy and ischemia}

\section{Neuropsychological assessment}


\chapter{Summary and Conclusions}


%\include{prelude} % Title page, nomenclature, abstract, table of contents, etc
%\include{specific_aims}                
%\include{background/background}
%\include{preliminary_results/preliminary_results}
%\include{research_design/research_design}
%\include{timeline/timeline}

%% Choose your bibliography style
%% plain is the basic style, others include ieeetr, siam, asm, etc
%\bibliographystyle{plain}
%% \altbibtitle{}
%\bibliography{preliminary_proposal}              % Make the bibliography

% \begin{appendices}               % Start of the Appendix Chapters.  If there is only
%                                  % one Appendix Chapter, then use \begin{appendix}
% \include{code}                   % Including computer code listings
% \include{bibref}                 % a BibTeX reference
% \include{math}                   % Complex Equations from the UW Math Department
% \include{acro}                   % A discussion on generating PDF files.
% \end{appendices}                 % End of the Appendix Chapters.  ibid on \end{appendix}
%\include{vita}                  % Optional Vita, use \begin{vita} vita text \end{vita}
\end{document}
